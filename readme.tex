\documentclass[A4]{ltxdoc}
\usepackage[mark]{gitinfo2}
\usepackage{markdown}   
\usepackage{fontawesome} 
\usepackage{hyperref, xcolor}
\usepackage{color}
\hypersetup{colorlinks=true, linkcolor=myblue, urlcolor=myblue,hyperindex}
\usepackage{url}
\usepackage[utf8]{inputenc}
\definecolor{myblue}{rgb}{0.22,0.45,0.70}
\renewcommand{\gitMarkPref}{\faGit}

\hypersetup{
   pdfauthor={Raphaël P. Barazzutti},
   pdftitle={itinfo2 and latexmk integration},
   pdfkeywords={gitinfo2;git;latexmk},
}

\begin{document}
\thispagestyle{plain}
\title{gitinfo2 and latexmk integration}
\renewcommand{\gitMark}{
Release:\gitReln{}: \gitAbbrevHash{} (\gitAuthorDate)}
\author{%
  Raphaël P. Barazzutti\thanks{%
    URL: \href{https://github.com/rbarazzutti/gitinfo2-latexmk}{github.com/rbarazzutti/gitinfo2-latexmk};
    Licence: \href{http://latex-project.org/lppl/lppl-1-3c.txt}{LaTeX Project Public License, version 1.3c};
    E-mail: \href{mailto:raphael.barazzutti@unine.ch}{\tt raphael.barazzutti@unine.ch}.}}

\maketitle
\thispagestyle{plain}
\begin{abstract}
    This tiny piece of software allows a painless and seamless integration of \textsf{gitinfo2} with \textsf{latexmk}. Instead of relying on \textsf{git}'s action hooks, this piece of code is loaded
    by \textsf{latexmk} when a build is triggered.
\end{abstract}

\markdownInput{README.md}


\end{document}